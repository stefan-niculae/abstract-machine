\documentclass[11pt]{report}

\usepackage{listings}
\usepackage{color}
\usepackage{amsmath}
\usepackage{bm}

\pagenumbering{gobble}  % no page number

\definecolor{grey}{RGB}{150,150,150}

% http://tex.stackexchange.com/questions/102299/automatic-line-break-for-writing-context-free-grammars
\lstset{
%  basicstyle=\itshape,
  xleftmargin=1em,
  keywordstyle=\itshape,
morekeywords={a,int,float,true,false,string,if,then,else,while,do,break!,continue!,exit!,z},
  literate={->}{{ \textcolor{grey}{$\rightarrow$} }}{2}
               {|}{{ \textcolor{grey}{$|$} }}{1}
               {<<}{{ \textcolor{grey}{$\langle$} }}{1}
               {>>}{{ \textcolor{grey}{$\rangle$} }}{1}
}

\begin{document}

\thispagestyle{empty}
\section*{Syntax}

Non-terminals:
  \begin{lstlisting}
  Expr -> nr | Expr iop Expr | var | ( E )
  Cond -> bl | Expr bop Expr 
  Stmt -> () | Stmt ; Stmt | var = Expr
        | if Cond then Body else Body
        | while Cond do Body
        | continue! | break! | exit!
  Body -> Stmt | { Stmt }
  \end{lstlisting}

\vspace{3em}
\noindent Terminals:
    \begin{lstlisting}
   iop -> +  | -  | * | /
   bop -> == | != | < | > | <= | >=
   var -> <<string>>
    nr -> <<int>> | <<float>>
    bl -> true  | false
  \end{lstlisting}


\noindent A program is represented by a statement (probably sequenced).




\newpage
\section*{Operational Semantics}

Initial configuration of executing statement $S$: $ \langle S, \emptyset, \emptyset \rangle $. Repeatedly apply these rules until there are no commands left ($c=\emptyset$).

\vspace{2em}
\noindent Arithmetic expressions $(E)$:
\begin{align*}
\langle n  \bm{c},  \bm{s},  \bm{m} \rangle  &\rightarrow   \langle  \bm{c}, n  \bm{s},  \bm{m} \rangle \\
\langle v  \bm{c},  \bm{s},  \bm{m} \rangle    &\rightarrow   \langle  \bm{c},  m(v),  \bm{s},  \bm{m} \rangle \\
\langle (E_1\ iop \ E_2)  \bm{c},  \bm{s},  \bm{m} \rangle    &\rightarrow   \langle E_1\  E_2\  iop  \ \bm{c},  \bm{s},  \bm{m} \rangle \\
\langle iop \ \bm{c}, n_2 n_1  \bm{s},  \bm{m} \rangle    &\rightarrow   \langle  \bm{c}, n  \bm{s},  \bm{m} \rangle  \ \text{where} \ n = n_1\ \underline{iop} \ n_2
\end{align*}

Execution will be guarded against division by zero.




\vspace{2em}
\noindent Boolean conditions $(C)$:
\begin{align*}
\langle b  \bm{c},  \bm{s},  \bm{m} \rangle    &\rightarrow   \langle b, n  \bm{s},  \bm{m} \rangle \\
\langle (E_1 \ bop \ E_2)  \bm{c},  \bm{s},  \bm{m} \rangle    &\rightarrow   \langle E_1\ E_2\ bop \ \bm{c},  \bm{s},  \bm{m} \rangle \\
\langle bop \  \bm{c}, n_2 n_1  \bm{s},  \bm{m} \rangle    &\rightarrow   \langle  \bm{c}, b  \bm{s},  \bm{m} \rangle \  \text{where} \ b = n_1\ \underline{bop}\ n_2
\end{align*}


\vspace{2em}
\noindent Statements $(S)$:
\begin{align*}
\langle ()  \bm{c},  \bm{s},  \bm{m} \rangle    &\rightarrow   \langle  \bm{c},  \bm{s},  \bm{m} \rangle \\
\langle (S_1 ; S_2)  \bm{c},  \bm{s},  \bm{m} \rangle    &\rightarrow   \langle S_1 S_2  \bm{c},  \bm{s},  \bm{m} \rangle \\
\langle v=E  \bm{c},  \bm{s},  \bm{m} \rangle    &\rightarrow   \langle E\ save\  \bm{c}, v  \bm{s},  \bm{m} \rangle \\
\langle save \  \bm{c}, n v  \bm{s},  \bm{m} \rangle    &\rightarrow   \langle  \bm{c},  \bm{s},  \bm{m}[v=n] \rangle \\
\langle continue \  \bm{c}, \bm{s},  \bm{m} \rangle    &\rightarrow   \langle  \bm{c'},  \bm{s},  \bm{m} \rangle \\
\langle break \  \bm{c}, \bm{s},  \bm{m} \rangle    &\rightarrow   \langle  \bm{c''},  \bm{s},  \bm{m} \rangle \\
\langle exit \  \bm{c}, \bm{s},  \bm{m} \rangle    &\rightarrow   \langle  \emptyset,  \emptyset, \emptyset \rangle
\end{align*}

where $\bm{c'} :=$  the first $entered\ while$ in $\bm{c}$ and the commands after it; and $\bm{c''} :=$ the commands after the first $entered\ while$ (excluding it).


\newpage
\noindent Branching $(if)$:
\begin{align*}
\langle (if\ C\ then\ S_t\ else\ S_f)  \bm{c},  \bm{s},  \bm{m} \rangle    &\rightarrow   \langle C\ branch\  \bm{c}, S_t S_f  \bm{s},  \bm{m} \rangle \\
\langle branch\  \bm{c}, true\  S_t S_f  \bm{s},  \bm{m} \rangle    &\rightarrow   \langle S_t  \bm{c},  \bm{s},  \bm{m} \rangle \\
\langle branch\  \bm{c}, false\ S_t S_f  \bm{s},  \bm{m} \rangle    &\rightarrow   \langle S_f  \bm{c},  \bm{s},  \bm{m} \rangle
\end{align*}


\vspace{2em}
\noindent Looping $(while)$:
\begin{align*}
\langle (while\ C\ do\ S)  \bm{c},  \bm{s},  \bm{m} \rangle    &\rightarrow  \langle C\ loop\  \bm{c}, C S  \bm{s},  \bm{m} \rangle \\
\langle loop\  \bm{c}, false\ C S  \bm{s},  \bm{m} \rangle    &\rightarrow   \langle  \bm{c},  \bm{s},  \bm{m} \rangle \\
\langle loop\  \bm{c}, true\  C S  \bm{s},  \bm{m} \rangle    &\rightarrow   \langle S (while\ C\ do\ S)  \bm{c},  \bm{s},  \bm{m} \rangle 
\end{align*}

When applying the last rule, we also mark the $while$ statement as $entered$. That is because encountering a loop with its condition evaluating to $true$ causes us to enter it (at least one execution of its $body$). The $entered$ flag is used for $break$ and $continue$ statements.

Execution will be guarded against infinite cycles.

\end{document}
